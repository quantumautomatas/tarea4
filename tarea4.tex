%Margenes, idioma y tipo de documento
\documentclass{article}
\usepackage[spanish]{babel}
\usepackage[utf8]{inputenc}
\usepackage[margin = 1.5cm]{geometry}
\usepackage[shortlabels]{enumitem}

%Algunas símbolos matemáticos (letras caligráficas)
\usepackage{amssymb}
\usepackage{amsmath}


\begin{document}
    
    %Titulo
    \title{Autómatas y Lenguajes formales 2019-2\\
    \large Ejercicio Semanal 4}

    \date{Fecha de entrega: 22 de febrero del 2019}

    \author{Sandra del Mar Soto Corderi\\
    Edgar Quiroz Castañeda}

    \maketitle

    %Problemas

    \begin{enumerate}
        \item {
        Demuestra que el operador de derivada preserva equivalencias, es decir si
		$\alpha \ = \ \beta$, entonces $\partial_{a} \alpha \ = \ \partial_{a} \beta$.\\

            
        }
        
        \item{
        Calcula la derivada de las expresiones regulares en cada inciso.\\
		\begin{enumerate}[a)]
		
		\item{
		$\partial{bb} (a^* \ + \  (a^*ba^*ba^*)^*)$\\
		
		
		}
		
		\item{
		$\partial{ab} ((a^*(baa)^*a^*)^*)$\\
		
		
		}
		
		\item{
		$\partial{a} ((aa \  + \  bb)^*)$\\
		
		
		}		
		
		\end{enumerate}		        
        
        
        }
    \end{enumerate}

\end{document}